\documentclass[10pt]{article}
%%%%%%%%%%%%%%%%%%%%%%%%%%%%%%%%%%%%%%%%
\usepackage{amsmath}
\usepackage{verbatim}
\usepackage[usenames,dvipsnames]{color}
\usepackage{ulem}
\usepackage{setspace}
\usepackage{lscape}
\usepackage{longtable}
\usepackage[top=1.25in,bottom=1.5in,left=1in,right=1.5in,landscape]{geometry}
\usepackage{graphicx}
\usepackage{epstopdf}
\usepackage[usenames,dvipsnames]{pstricks}
\usepackage{epsfig}
\usepackage{pstricks-add}
\usepackage{pst-node}
\usepackage{fancyhdr}
\usepackage[absolute,showboxes]{textpos}

%TCIDATA{OutputFilter=LATEX.DLL}
%TCIDATA{Version=5.00.0.2552}
%TCIDATA{<META NAME="SaveForMode" CONTENT="1">}
%TCIDATA{Created=Thursday, August 28, 2003 13:38:44}
%TCIDATA{LastRevised=Thursday, August 14, 2008 15:20:27}
%TCIDATA{<META NAME="GraphicsSave" CONTENT="32">}
%TCIDATA{<META NAME="DocumentShell" CONTENT="Standard LaTeX\Blank - Standard LaTeX Article">}
%TCIDATA{Language=American English}
%TCIDATA{CSTFile=LaTeX article (bright).cst}

\setcounter{MaxMatrixCols}{10}

\newenvironment{proof}[1][Proof]{\noindent\textbf{#1.} }{\ \rule{0.5em}{0.5em}}
\setlength{\columnsep}{.2in}

\renewcommand{\labelitemii}{$\cdot$}

\pagestyle{fancy} \fancyhead{} \fancyfoot{} \rfoot{} \lfoot{}

\newcommand{\slide}[2]{
\begin{textblock}{11}(0,0)
\textcolor{Black}{\textbf{\huge \rule{0pt}{1in} \raisebox{.2in}{#1}}}
\end{textblock}
\begin{Large} \noindent
#2
\end{Large}
\vfill \pagebreak}

\setlength{\TPHorizModule}{1in}
\setlength{\TPVertModule}{1in}
\textblockcolour{Yellow}
\renewcommand{\headrulewidth}{0pt}



\begin{document}
\onehalfspacing 

\lfoot{The Romer Model} \rfoot{Economic Growth}

\slide{Economics of Technology}{Solow model with
\begin{equation}
\frac{\dot{A}}{A} = \theta \frac{(s_R L)^{\lambda}}{A^{1-\phi}}
\end{equation}
to describe growth of $A$ gives us a way of describing trend growth as a function of population growth.

\vspace{.25in}\noindent But what determines $s_R$? What determines how much effort the economy puts into R\&D? Romer model will provide answers. Has three components
\begin{itemize}
	\item Final goods sector - competitive
	\item Intermediate goods sector - imperfect competition
	\item Research sector - comes up with new ideas for intermediate goods
\end{itemize}
}

\slide{Final Goods Sector}{Final goods are produced by perfectly competitive firms using this function
\begin{equation}
Y = L_Y^{1-\alpha} \int_0^A x_j^{\alpha} dj
\end{equation}
\begin{itemize}
	\item $L_Y$ is the number of workers in the final goods sector
	\item $A$ indexes the number of intermediate goods used
	\item $x_j$ is the amount of int. good $j$ used by the final goods sector
\end{itemize}

\vspace{.25in}\noindent Final goods firms are using lots of intermediate goods and essentially adding them up like this:
\begin{equation}
\int_0^A x_j^{\alpha} dj \approx \sum^A_0 x_j^{\alpha} = x_0^{\alpha} + x_1^{\alpha} + ... + x_A^{\alpha}
\end{equation}
}

\slide{Final Goods Firms}{The final goods firms try to maximize profits
\begin{equation}
max_{L_Y,x_j} L_Y^{1-\alpha} \int_0^A x_j^{\alpha} dj - wL_Y - \int_0^A p_j x_j dj
\end{equation}
which gives the first-order conditions
\begin{equation}
w = (1-\alpha)\frac{Y}{L_Y}
\end{equation}
and \begin{equation}
p_j = \alpha L_Y^{1-\alpha} x_j^{\alpha-1}
\end{equation}
for each individual intermediate good.

\vspace{.25in}\noindent Final goods firms are just setting marginal cost (wage, price of good $j$) equal to the marginal product of the input. 
}

\slide{Intermediate Goods Firms}{The int. goods come from separate firms. The int. goods firms are monopolists - only one firm can produce each separate int. good (patents, tacit knowledge, branding, etc..)

\vspace{.25in}\noindent Int. good firms produce those int. goods by transforming capital into them one-for-one. One unit of capital produces one unit of int. goods.

\vspace{.25in}\noindent The profits of any given int. good firm $j$ are
\begin{equation}
\pi_j = p_j(x_j)x_j - r x_j
\end{equation}
where $p_j(x_j)$ is the demand function of the final goods firm for the intermediate good $j$. 

\vspace{.25in}\noindent Because the firm is a monopolist, they take into account how demand changes as they produce more $x_j$ - that is, they know the price $p_j$ that final good firms will pay for any given output $x_j$.
}

\slide{Profit Maximizing}{Profits are
\begin{equation}
\pi_j = p_j(x_j)x_j - r x_j.
\end{equation}
Firm pick $x_j$ to maximize profits.

\vspace{.25in}\noindent First-order condition
\begin{equation}
p'(x)x + p(x) = r
\end{equation}
which says that marginal revenue equals marginal cost. Re-arrange to
\begin{equation}
p'(x)\frac{x}{p} + 1 = \frac{r}{p}
\end{equation}
or 
\begin{equation}
p = \frac{1}{1+\frac{p'(x)x}{p}}r.
\end{equation}
}

\slide{Firm Markups}{The firms decision is to set price as follows,
\begin{equation}
p = \frac{1}{1+\frac{p'(x)x}{p}}r.
\end{equation}
The term $p'(x)x/p$ is the elasticity of demand. What is this elasticity? We know the demand function for an intermediate good $j$,
\begin{equation}
p_j = \alpha L_Y^{1-\alpha} x_j^{\alpha-1}
\end{equation}
so the elasticity is $\alpha - 1$. Thus the firm charges
\begin{equation}
p = \frac{1}{1+\alpha-1}r = \frac{r}{\alpha}.
\end{equation}
}

\slide{Firm Markups}{The int. good firm charges 
\begin{equation}
p = \frac{r}{\alpha}
\end{equation}
for its output. Thus price ($p$) is greater than marginal cost ($r$), as $\alpha<1$. 

\vspace{.25in}\noindent Intermediate good firms thus earn profits. These profits are what will be the reason for innovation. The opportunity to earn profits will incent people to try and start int. good firms.

\vspace{.25in}\noindent Note that the profit-maximization problem is identical for all int. good firms. All of them charge the same price. Given that the demand function is identical for all int. goods, this means that the final goods firm buys the same amount of all of them.
\begin{equation}
x_j = x
\end{equation}
}

\slide{Aggregate Output}{Putting this all together. The total amount of int. goods produced must be equal to the capital stock, as each int. good firm uses capital to produce, so
\begin{equation}
K = \int_0^A x_j dj
\end{equation}
and because all $x_j = x$, it must be that
\begin{equation}
x = \frac{K}{A}.
\end{equation}

\vspace{.25in}\noindent Final output is
\begin{equation}
Y = L_Y^{1-\alpha}\int_0^A x_j^{\alpha} dj = L_Y^{1-\alpha} A x^{\alpha}
\end{equation}
which when we plug in $x = K/A$ becomes
\begin{equation}
Y = K^{\alpha} (AL_Y)^{1-\alpha}.
\end{equation}
Aggregate final output is described just like we did with the Solow model. Only the underlying market structure is different.
}

\slide{Distribution of Income}{From the final goods firms first-order condition we know that
\begin{equation}
w L_Y = (1-\alpha)Y,
\end{equation}
which means that 
\begin{equation}
\int_0^A p_j x_j dj = \alpha Y
\end{equation}
or the total revenues of int. goods firms add up to $\alpha$ of final output. 

\vspace{.25in}\noindent We know that $p_j = r/\alpha$ for all int. good firms, and $x_j = K/A$. So plugging in we get that
\begin{equation}
\int_0^A \frac{r}{\alpha}\frac{K}{A} dj = \alpha Y
\end{equation}
which solves down to
\begin{equation}
rK = \alpha^2 Y.
\end{equation}
}

\slide{Profits}{Total profits of all int. good firms are
\begin{eqnarray}
\int_0^A \pi_j dj &=& \int_0^A p_j x_j dj - \int_0^A r x_j dj \\
 &=& \alpha Y - A x r  \\
 &=& \alpha Y - A \frac{K}{A}\frac{\alpha^2 Y}{K} \\
 &=& \alpha Y - \alpha^2 Y \\
 &=& \alpha(1-\alpha)Y.
\end{eqnarray}

\vspace{.25in}\noindent For any individual int. good firm, since they are identical, profits are
\begin{equation}
\pi_j = \frac{\alpha(1-\alpha)Y}{A}
\end{equation}
or each firm earns $1/A$ of the total profits accruing to int. good firms in the economy.
}

\slide{Imperfect Competition and Profits}{Note that total output is distributed across three types of payments
\begin{equation}
rK + wL_Y + \int_0^A \pi_j dj = \alpha^2 Y + (1-\alpha)Y + \alpha(1-\alpha)Y = Y
\end{equation}
Not all of output is paid to capital and labor. Some part of output goes to profits. This is because we have imperfect competition (monopolistic competition, to be exact). There are true profits being earned - which will be the incentive for new innovations.

\vspace{.25in}\noindent Compare to standard Solow model with
\begin{equation}
rK + wL = \alpha Y + (1-\alpha)Y = Y
\end{equation}
where all of output is paid to either capital or labor. Nothing is left to compensate innovators/inventors, so there is no incentive to innovate. 

}

\slide{Research Sector}{The last sector is research. Individuals can work at trying to invent a new type of intermediate good. 
\begin{itemize}
	\item For each researcher, they take the chance of discovering a new idea as given, and equal to $\overline{\theta}$
	\item $\overline{\theta} = \theta L_A^{\lambda-1} A^{\phi}$, as in our mechanical model, but researchers do not take this into account
	\item If a researcher discovers a new idea, they receive a patent on that idea. They own the rights to produce that new intermediate good
	\item The researcher can then sell that patent to a firm that wants to be the monopolist producing that int. good (or the researcher could start their own firm)
	\item We assume that people can move back and forth from research to working in the final goods sector, so the return to research (the expected value of getting a patent) will be equal to the return to working (the wage)
\end{itemize}
}

\slide{Patent Value}{We use an arbitrage equation to get the value of a patent, $P_A$
\begin{equation}
r P_A = \pi + \dot{P}_A
\end{equation}
\begin{itemize}
	\item $rP_A$ is the amount earned from investing $P_A$ dollars in the bank. This is the alternative to buying the patent
	\item $\pi$ is the profits earned from owning the patent, solved for above
	\item $\dot{P}_A$ is the change in the value of the patent if you own it - you could resell it later for a profit (or loss)
\end{itemize}

\vspace{.25in}\noindent The arbitrage equation says that the return to owning the patent (the right-hand side) must be equal to the return to simply investing the same dollars in the bank (the left-hand side). If not, one could make a profit.

}

\slide{Balanced Growth Path for Patents}{Re-arrange the arbitrage equation to
\begin{equation}
r = \frac{\pi}{P_A} + \frac{\dot{P}_A}{P_A}.
\end{equation}
Along the BGP, we know that $r$ is constant. So therefore the ratio $\pi/P_A$ must be constant, and $\dot{P}_A/P_A$ must be constant. 

\vspace{.25in}\noindent Along the BGP, we know that 
\begin{equation}
\pi = \alpha(1-\alpha)\frac{Y}{A} = \alpha(1-\alpha)\frac{Y}{L}\frac{L}{A}.
\end{equation}
Along the BGP, $Y/L$ grows at the rate $g$. $A$ grows at the rate $g$. $L$ grows at the rate $n$. So $\pi$ must grow at the rate $n$. 

\vspace{.25in}\noindent This implies that $\dot{P}_A/P_A = n$, to keep $\pi/P_A$ constant. Putting that into the arbitrage equation gives
\begin{equation}
r = \frac{\pi}{P_A} + n
\end{equation} 
or
\begin{equation}
P_A = \frac{\pi}{r-n}
\end{equation}
which says that the value of a patent is the present discounted value of the profits from owning the patent.
}

\slide{Labor Market Equilibrium}{The last thing to do is solve for the fraction of workers who do research. People can work or do research. Doing research earns
\begin{equation}
w_R = \overline{\theta}P_A
\end{equation}
or the chance of making a discovery times the value of a discovery.

\vspace{.25in}\noindent Working earns
\begin{equation}
w_Y = (1-\alpha)\frac{Y}{L_Y}
\end{equation}
which is just the wage paid by final goods sectors.

\vspace{.25in}\noindent Set these equal
\begin{equation}
w_R = w_Y
\end{equation}
meaning that people will move back and forth until this holds.
}

\slide{Solving for Equilibrium}{With
\begin{equation}
w_R = w_Y
\end{equation}
we have
\begin{eqnarray}
\overline{\theta}P_A &=& (1-\alpha)\frac{Y}{L_Y} \\
\overline{\theta}\frac{\pi}{r-n} &=& (1-\alpha)\frac{Y}{L_Y} \\
\overline{\theta}\frac{\alpha(1-\alpha)}{r-n}\frac{Y}{A} &=& (1-\alpha)\frac{Y}{L_Y} \\
\frac{\alpha}{r-n}\frac{\overline{\theta}}{A} &=& \frac{1}{L_Y}
\end{eqnarray}
}

\slide{Solving for Equilibrium}{
\begin{equation}
\frac{\alpha}{r-n}\frac{\overline{\theta}}{A} = \frac{1}{L_Y}
\end{equation}
Now, we need to use the mechanical relationship that $\dot{A}/A = \overline{\theta}L_A/A$. Along a BGP $\dot{A}/A = g$, so along a BGP $\overline{\theta}/A = g/L_A$.
\begin{eqnarray}
\frac{\alpha}{r-n}\frac{g}{L_A} &=& \frac{1}{L_Y} \\
\frac{\alpha g}{r-n} &=& \frac{L_A}{L_Y} \\
\frac{\alpha g}{r-n} &=& \frac{s_R}{1-s_R}
\end{eqnarray}
which you can solve for
\begin{equation}
s_R = \frac{1}{1 + \frac{r-n}{\alpha g}}
\end{equation}
}

\slide{Equilibrium $s_R$}{We have the solution for how many people will do research
\begin{equation}
s_R = \frac{1}{1 + \frac{r-n}{\alpha g}}.
\end{equation}
\begin{itemize}
	\item The discount rate $r-n$ has a negative effect on $s_R$. If the discount rate goes up, research effort goes down. Higher discount rates mean lower PDV's of patents, and so doing R\&D is not as lucrative
	\item The growth rate $g$ has a positive effect on $g$. If the growth rate is high, this is because people are finding new ideas quickly. Therefore doing research is very likely to lead to a patent, so more people want to do research
\end{itemize}
}

\slide{Optimal $s_R$?}{Is the share $s_R$ optimal in the sense that it maximizes output per worker along the BGP? In general, no. Recall that
\begin{equation}
y(t) = \frac{s_R L(t)}{g} (1-s_R) \left(\frac{s}{\delta +n +g} \right)^{\alpha/(1-\alpha)}
\end{equation}
along the BGP. There are conflicting effects of $s_R$, and there is some value of $s_R$ that gives us the highest level of $y(t)$. 

\vspace{.25in}\noindent Why does the market not necessarily get us to the optimal value of $s_R$? 
\begin{itemize}
	\item Value of patents does not capture the effect of a new idea on the future flow of ideas. I may either lower (if $\phi<0$) or raise (if $\phi>0$) the productivity of other researchers by inventing a new idea (raising $A$). If $\phi>0$, the market under-provides research.
	\item Value of patents does not capture the duplication of effort. With $\lambda>0$ my research is slowing down your research, and the market over-provides research.
	\item Imperfect competition. The incentive to innovate is based on profits only, not the total value of the new int. good to producing final goods. So the market tends to under-provide research. Some estimates that R\&D has a social, or total, return of 40-60\%, exceeding any private return on investment in R\&D from added profits.
\end{itemize}
}


\end{document}